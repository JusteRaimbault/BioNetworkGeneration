\documentclass{beamer}
  \usepackage[english]{babel}
  \usepackage[utf8]{inputenc}
  \usepackage{times}
  \usepackage{amsmath,amsthm, amssymb, latexsym,ragged2e}
  \boldmath
  \usepackage{multicol}
  
  \usetheme{Poster}
  \usepackage[orientation=portrait,size=a0,scale=1.4]{beamerposter}



% Abstract

% Nous implémentons un modèle de développement d'un organisme de type slime mould, qui a été introduit en biologie, et l'appliquons à la conception des réseaux de transport. À partir d'un réseau potentiel couvrant l'espace et d'une distribution d'origines et destinations, l'algorithme renforce itérativement les liens les plus fréquentés, pour converger vers un réseau auto-organisé adapté à la distribution des trajectoires de mobilité. Son application est illustrée pour la conception d'un réseau de ligne de bus et d'un réseau routier, sur des cas d'étude réels. Nous démontrons par ailleurs la capacité du modèle à générer un ensemble de réseaux optimaux au sens de Pareto pour les deux objectifs contradictoires de robustesse et de coût, par une exploration systématique par l'intermédiaire du logiciel OpenMOLE, sur un système métropolitain polycentrique stylisé



  \title{Des systèmes naturels aux systèmes urbains: génération de réseaux de transport optimaux par modèle \emph{slime-mould}}
  \author[juste.raimbault@polytechnique.edu]{J. Raimbault$^{1,2}$}
  \institute[]
  {$^1$ UPS CNRS 3611 ISC-PIF et $^2$ UMR CNRS 8504 Géographie-cités\vspace{1cm}}
  \date{}
  
  \logo{
  \hfill
  \includegraphics[height=8cm,width=0.65\columnwidth]{figures/isc}
  \includegraphics[height=8cm,width=0.29\columnwidth]{figures/geocite}
  \hfill\hfill
  }


  %%%%%%%%%%%%%%%%%%%%%%%%%%%%%%%%%%%%%%%%%%%%%%%%%%%%%%%%%%%%%%%%%%%%%%%%%%%%%%%%%5
  \begin{document}
  \begin{frame}{} 
  
    \vfill
    \begin{columns}[t]
      \begin{column}{.49\textwidth}
      
      \vspace{-1cm}
      
        \begin{block}{Introduction}
        \vspace{-1cm}
        \begin{columns}[t]
        \begin{column}{.95\textwidth}
          \begin{itemize}         
          \item \justify Méthodes classiques de conception et d'évaluation des infrastructures de transport basées sur scenarios exogènes \cite{wegener2004land}; de type optimisation et/ou analyse de données \cite{karlaftis2011statistical}.
          \end{itemize}
          \bigskip
          \begin{itemize} 
          \item \begin{justify} Ingénierie morphogénétique à la croisée des systèmes auto-organisés et architecturés \cite{doursat2012morphogenetic} ; application démontrée à la conception d'infrastructures de transport \cite{bebber2007biological}.
          \end{justify}
          
          \bigskip
          \item \begin{justify}Application d'un modèle de croissance de \emph{slime-mould} à la conception multi-objectifs d'un réseau de transport.\end{justify}
          \end{itemize}
          \end{column}
          \end{columns}
        \end{block}
        
         \begin{block}{Modèle}
        %\vspace{-1cm}
        \begin{columns}[t]
        \begin{column}{.95\textwidth}
        \vspace{-2cm}
        \begin{justify}
          Modèle de croissance d'un \emph{slime-mould}~\cite{tero2010rules} : principe \emph{d'exploration puis renforcement}.
          %pour une moisissure à la recherche de ressources.
          
          \vspace{1.5cm}
                    
          $\rightarrow$ \textit{Etude de l'aspect renforcement :} réseau initial homogène de tubes $ij$, longueur $L_{ij}$, diamètre variable $D_{ij}$, traversés par un flux de fluide $Q_{ij}$. Sommets $i$ à la pression $p_i$. Un nombre de noeuds $N$ sont à desservir, parmi eux aléatoirement à chaque étape l'un est source $p_{i_+}=I_0$ et l'autre puits $p_{i_-}=-I_0$
          
          \vspace{1.5cm}
          
          
          $\rightarrow$ \textit{Itération du modèle :}
          \begin{enumerate}
          \item Détermination des flux par lois de Kirchoff (analogie électrostatique, résolution d'un système fermé) : loi d'Ohm
          \begin{equation}
Q_{ij}=\frac{D_{ij}}{L_{ij}}\cdot(p_{i}-p_{j})
\end{equation}
et conservation des flux
\begin{equation}
\sum_{j\rightarrow i}Q_{ij} = 0 , \sum_{j\rightarrow i_\pm}Q_{i_{\pm}j} = \pm I_0
\end{equation}



\item Evolution du diamètres de tubes ($\gamma$ paramètre de renforcement)
\begin{equation}
\frac{dD_{ij}}{dt}=\frac{\left|Q_{ij}\right|^{\gamma}}{1+\left|Q_{ij}\right|^{\gamma}}-D_{ij}
\end{equation}
          \end{enumerate}

\vspace{1.5cm}

$\rightarrow$ \textit{Extraction du réseau final} après convergence selon un paramètre de seuil de diamètre ou un nombre maximal d'itérations.
 %(distributions finales généralement bimodales).

%\bigskip 
%\textit{Modèle multi-échelle :} Dynamique des diamètres supposées constantes pendant l'itération pour obtenir les flux. 
%\bigskip 
%\textit{Implémentation en NetLogo ouverte~\cite{impl}}

\end{justify}
          \end{column}
          \end{columns}
        \end{block}
        
        
        \begin{block}{Indicateurs}
       \vspace{-2cm}
        \begin{columns}[t]
        \begin{column}{.95\textwidth}
        \begin{justify}
          Comportement du modèle évalué au travers d'indicateurs contradictoires de performance pour le réseau généré $(V_f,E_f)$:
          %, qui peuvent être vu comme des objectifs contradictoires :
          \bigskip
          \begin{itemize}
          \item Coût de construction $c=\sum_{ij\in E_f}D_{ij}(t_f)$
          \bigskip
          \item \begin{justify}Performance moyenne~\cite{banos2012towards}
          \[
          v=\frac{1}{|V_f|^2}\sum_{i,j\in V_f}\frac{d_{i\rightarrow j}}{||\vec{i}-\vec{j}||}
          \]
          \end{justify}
          \bigskip
          \item Robustesse: indice \textit{Network Trip Robustness}, impact de la suppression des liens~\cite{sullivan2010identifying}
          \end{itemize}

          \end{justify}
          \end{column}
          \end{columns}
        \end{block}
        
        \begin{block}{Exploration du modèle}
        \begin{columns}[t]
        \begin{column}{.6\textwidth}
        	\begin{justify}
        	\vspace{-6cm}
        	Exploration de l'espace des paramètres du modèle par calcul intensif, rendu transparent par le logiciel libre d'exploration de modèles OpenMOLE \cite{reuillon2013openmole}
        	\end{justify}
        \end{column}
		\begin{column}{.35\textwidth}
			\centering
        	\includegraphics[width=\columnwidth]{figures/openmole4.png}
        \end{column}
        \end{columns}
        \end{block}

        
        
%        \begin{block}{Analyse de Sensibilité}
%       \vspace{-1cm}
%        \begin{columns}[t]
%        \begin{column}{.47\textwidth}
%                    
%          \includegraphics[width=0.5\columnwidth,height=8cm]{figures/graphe_cout}
%          \includegraphics[width=0.5\columnwidth,height=8cm]{figures/graphe_NTR}\\
%          \bigskip
%          \textit{Sensibilité des indicateurs aux paramètres $(N,I_0)$.}
%          
%
%          \end{column}
%          \begin{column}{.47\textwidth}
%          
%           
%            \includegraphics[width=0.5\columnwidth,height=6cm]{figures/networkDense}
%          \includegraphics[width=0.5\columnwidth,height=6cm]{figures/networkLessDense}\\
%          \bigskip
%          \begin{justify}
%          \textit{Sensibilité de la topologie du réseau au coefficient de renforcement $\gamma$. Gauche : $\gamma \sim 1$, réseau robuste. Droite : $\gamma >> 1$, réseau arborescent.}
%          \end{justify}
%          \end{column}
%          \end{columns}
%        \end{block}
%        
        
      \end{column}
      
      
      
      \begin{column}{.49\textwidth}
      
      \vspace{-1cm}
      
       \begin{block}{Application : desserte optimale}
        \vspace{-1.5cm}
          \begin{columns}[t]
        \begin{column}{.95\textwidth}
        %Mission de prospective pour la mairie de Romainville : itinéraire d'une navette intra-urbaine avec points de desserte imposés.\end{justify}
        % , mais multi-objectif (coût, vitesse, robustesse)
       \begin{justify} Problème type voyageur de commerce multi-objectifs : itinéraire de desserte pour une navette intra-urbaine avec points de passage imposés, et objectifs de coût et efficacité.
       \end{justify}
         
         % la génération de réseau \emph{bottom-up} par application du modèle sur un réseau initial construit par données géographiques réelles (réseau de rues) donne une solution proche du front de Pareto.
         %\bigskip\bigskip
          \vspace{1cm}
          
          {
          %\centering
          \includegraphics[width=0.33\columnwidth]{figures/tick1}\vrule width 1pt
          \includegraphics[width=0.33\columnwidth]{figures/tick10}\vrule width 1pt
          \includegraphics[width=0.33\columnwidth]{figures/tick20}\\\hrule height 1pt
          \includegraphics[width=0.33\columnwidth]{figures/tick50}\vrule width 1pt
          \includegraphics[width=0.33\columnwidth]{figures/tick101}\vrule width 1pt
          \includegraphics[width=0.33\columnwidth]{figures/reseauFinal}\\
          \bigskip
          \small\textit{Convergence progressive du réseau vers le réseau optimal desservant les points fixés (en rouge), en partant d'un réseau initial à diamètres égaux (réseau de rues).}\bigskip
          
          }
          \end{column}
          \end{columns}
        \end{block}
      
      
      
        \begin{block}{Application : réseaux métropolitains}
          \begin{columns}[t]
        \begin{column}{.95\textwidth}
        \begin{justify}
        \vspace{-2cm}
          
          
          Dans le cadre d'une configuration métropolitaine polycentrique stylisée \cite{le2015modeling}, comment élaborer automatiquement différents scénarios pour un nouveau réseau de transport ?
          
          \vspace{1.5cm}
          
          {\color{white}
          \centering
          \includegraphics[width=0.33\columnwidth]{figures/bionw_territ6_gamma1_1_bis.png}\vrule width 1pt
          \includegraphics[width=0.33\columnwidth]{figures/bionw_territ6_gamma1_2_quart.png}\vrule width 1pt
          \includegraphics[width=0.33\columnwidth]{figures/bionw_territ6_gamma1_25_bis.png}\\\hrule height 1pt
          \includegraphics[width=0.33\columnwidth]{figures/bionw_territ6_gamma1_3_bis.png}\vrule width 1pt
          \includegraphics[width=0.33\columnwidth]{figures/bionw_territ6_gamma1_5.png}\vrule width 1pt
          \includegraphics[width=0.33\columnwidth]{figures/bionw_territ6_gamma1_8.png}\\
          \bigskip
          \color{black}
          \small\textit{Réseaux stylisés obtenus pour des valeurs décroissantes de $\gamma$, pour une même configuration des centres de population à desservir (densité de population générée par mélange d'exponentielles).}
          }
          
          \vspace{1.5cm}
          
          {
          \centering
          %\hspace{-1cm}
          \includegraphics[width=0.5\columnwidth]{figures/pareto_cost-robustness_facetCenterNumber_withgaussianCI.png}
          \includegraphics[width=0.5\columnwidth]{figures/pareto_cost-speed_facetCenterNumber_withgaussianCI.png}\\
         
         \bigskip
          
          \small
          \textit{Optimisation de Pareto : fronts bi-objectifs entre les couples d'indicateurs obtenus pour un échantillonnage de 1000 valeurs pour $\gamma$, nombre de centres de 3 à 6 (sous-graphes) et 100 réplications (barres d'erreurs : intervalles de confiance à 95\%).}
         }
          
          \end{justify}
          \end{column}
          \end{columns}
        \end{block}


%        \begin{block}{Conclusion}
%         \begin{columns}[t]
%        \begin{column}{.95\textwidth}
%          
%          \end{column}
%          \end{columns}
%        \end{block}
        
        
        
      \end{column}
    \end{columns}
    
    %\begin{columns}[t]
    %  \begin{column}{\linewidth}
    
    \begin{block}{References}
        {\tiny
        \vspace{-1cm}
        \begin{multicols}{4}
          \bibliographystyle{apalike}
          \bibliography{/Users/juste/ComplexSystems/CityNetwork/Biblio/Bibtex/CityNetwork,biblio}
        \end{multicols}
          }
        \end{block}
    %\end{column}
    %\end{columns}
    
  \end{frame}
\end{document}





